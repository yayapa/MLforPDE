Differential equations are a ubiquitous tool for modeling the processes in physics, chemistry, finance, etc. Their numerical solution remains often challenging due to infeasible computational costs. In contrast, the application of machine learning shows great results in many related areas. This seminar work addresses current challenges in solving partial differential equations and investigates the diversity of the recent approaches that use machine learning techniques. The state-of-the-art works are explored by categorizing them into three major directions. First, theory-guided approaches that integrate the underlying physics of the process into machine learning models. Second, neural operators that directly learn a mapping between functional parameters to the solutions facilitating a high level of generalization. Third, high-dimensional approaches that mitigate the curse of dimensionality and propose effective solvers for equations with more than 100 dimensions.

\keywords{PDE \and Machine Learning \and Deep Learning}