\section{Introduction}
Many problems in physics, finance, engineering are described through differential equations. Differential equations provide a natural language to model processes in real world. Nowadays computers usually solve them numerically. However, traditional numerical methods require a lot of computational power and become infeasible, since it is desired to simulate complex processes precisely. Thus, applied mathematics is permanently searching for the new numerical methods or ''effective equations`` that are derived from the original differential equation, but can be solved faster with satisfactory accuracy.

On the other hand, in recent years, machine learning and, especially, deep learning have achieved great results in various areas. Therefore, these techniques have been also applied to solve the differential equations. This seminar work aims at broad investigating of the numerous approaches used to solve partial differential equations. We point out three main directions of recent work. First, theory-guided approaches that try incorporating physics prior knowledge into learning models, to make them more understandable and get rid of the black-box nature of deep learning. Second, neural operators that learn the solution operator, instead of learning of a concrete instance and, thus, promote generalization. Third, the distinct group of approaches for solving differential equations in high dimensional spaces, which notoriously suffer from the curse of dimensionality. In this work, we mainly focus on the applied machine learning techniques and not on the mathematical or physical meanings.

The remainder of this work is structured as follows. \Cref{sec:foundations} overviews the necessary fundamentals in numerical solving of partial differential equations. In \cref{sec:tga} we discuss the diversity of theory-guided approaches. In \cref{sec:neuraloperators} we present the novel neural operators applied to solve differential equation. \Cref{sec:highdim} describes the approaches for solving high-dimensional differential equations. In \cref{sec:conclusion} we summarize the conclusions of this seminar work and point out the possible direction of future works. 

